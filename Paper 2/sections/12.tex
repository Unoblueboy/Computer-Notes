\subsection{Functional programming paradigm}

  Before we get into this, the easiest way to understand most of this is to start learning a functional programming language, I suggest learning Haskell and using \textit{http://learnyouahaskell.com/chapters}

  \noindent
  \marginnote{4.12.1.1}within functional programming, each function which you define has a type. A functions type dictates what inputs it can take, and what it will output. For example, the function \verb|f :: a -> b| defines a function called \verb|f|, which takes an input of type \verb|a|, and outputs a value of type \verb|b|. in the above example, \verb|a| is the domain and \verb|b| is the co-domain. The following is somewhat implicit, but the domain is of type \verb|a| and the co-domain is of type \verb|b|. 

  \noindent
  \marginnote{4.12.1.2}A first-class object fulfils the following criteria, it can:
  \begin{itemize}
  	\item Appear in an expression
  	\item Be assigned to a variable
  	\item be assigned as an argument
  	\item be returned in function calls
  \end{itemize}
  
  In functional programming languages, and object oriented programs which support a function object, functions are first-class objects, this means they can be used as arguments of functions, as well as output of functions.

  \noindent
  \marginnote{4.12.1.3}Function application is the act of applying a function to its arguments, for example if we have the function (where A x A is the cartesian crossproduct of the set of real numbers with itself):
  \begin{verbatim}multFour :: (Num A) => A x A x A x A -> A
multFour w x y z = w * x * y * z\end{verbatim}
  Then \verb|multFour 2 3 4 5| would represent function application of the function add on the arguments 2, 3, 4, and 5 (which would produce 120)

  \noindent
  \marginnote{4.12.1.4}Partial function application is when you provide a function less arguments than it can take, meaning that a function is returned with the number of arguments that you left out. For example, if we were to define a function as follows
\begin{verbatim}multFour :: (Num a) => a -> a -> a -> a -> a
multFour w x y z = w * x * y * z\end{verbatim}
  Now if we were to  do \verb|multFour 3 4|, this is a partially applied function, as we have only supplied 2 of the 4 arguments needed, so two fully apply the function, two more arguments would needt to provided to fully apply the function. We can somewhat see this if we rewrite the type of the function in its full form. Its type would be:
\begin{verbatim}multFour :: (Num a) => a -> (a -> (a -> (a -> a)))\end{verbatim}
 And thus we see that this function is in fact a series of partial function applications, that when done 4 times, results in a number. When we normally write this out, we leave out brackets as they are unnecessary, and make the type look more confusing than is needed.
 

  \noindent
  \marginnote{4.12.1.5}Composition of functions, is the process of taking a function of a function, producing a new function, for example if we have $f(x)=x^5$ and $g(x)=2x+3$, then $f(g(x)) = f(2x+3) = (2x+3)^5$. Another way to represent $f(g(x))$ is $f \circ g$. To define composition in general using haskell:
  \begin{verbatim}composite :: (B -> C) -> (A -> B) -> A -> C
composite f g x = f (g x)\end{verbatim}

\subsection{Writing functional programs}

  \noindent
  \marginnote{4.12.2.1}As I said at the beginning you should learn how to at least do some basic programs in a functional programming language, not only to help you understand the concepts in this chapter, but also because it is a part of the specification.
  
  Here is all you need to know about higher order functions. It is a function that takes a function as an argument or returns a function as a result, or does both.
  
  There are three functions that are core to most functional programs that involve a list:
  \begin{itemize}
  	\item Map
	  	\subitem This is a higher order function which applies a provided function into a provided list of values, return a list of the results, for example, if you wanted to add 2 to [1,2,3,4,5], you could use: \verb|map (+2) [1,2..5]|, which would return \verb|[3,4,5,6,7]|
  	\item Filter
	  	\subitem This is another higher order function which is able to filter a list of values when provided a filtering function (it must return true of false and take only one argument which is the same type as the list provided), an example for a list of even numbers from 1 to 10:
  	\begin{verbatim}
even n = if n `mod` 2 == 0 then True else False
filter even [1,2..10]\end{verbatim}
  	\item Reduce/ Fold
	  	\subitem This higher order function reduces a list of values to a single value by repeatedly applying a combining function to the list of values, In haskell, there is \verb|foldl| and \verb|foldr|. They both work slightly differently, but in most cases, you will probably need \verb|foldl| (if you are seriously interested you can google the difference, it's to do with the order in which the function is applied). So for example, if we wanted to calculate $10!$, we could use \verb|foldl * 1 [1,2..10]|. The random 1 in the middle simply says what the starting value should be.
  \end{itemize}

\subsection{Lists in functional programming}

  \noindent
  \marginnote{4.12.3.1}Time for random things functional programs do with lists. Whenever you write a list you will most likely write it like [1,2,3,4,5]. In functional programming languages, a list can be written as a concatenation of its head (its first element) and its tail (the rest of the list), so for example, in haskell [1,2,3,4,5] = 1:[2,3,4,5] = 1:2:[3,4,5] = 1:2:3:[4,5] = 1:2:3:4:[5] = 1:2:3:4:5:[]. [1,2,3,4,5] is syntatic sugar for 1:2:3:4:5:[] (it represents the same thing, and the first is easier to read and write). As I said above, the head of the list is the first element of the list, and the tail is the rest of the list. A list can be empty and is represented by [].
  
  There are several operations you need to know:
  \begin{itemize}
  	\item return the head of a list
	  	\subitem \verb|head [1,2,3,4,5]| $\implies$ \verb|1|
  	\item return the tail of a list
	  	\subitem \verb|tail [1,2,3,4,5]| $\implies$ \verb|[2,3,4,5]|
  	\item test for empty list
	  	\subitem \verb|null [1,2,3,4,5]| $\implies$ \verb|False|, \verb|null []| $\implies$ \verb|True|
  	\item return length of a list
	  	\subitem \verb|length [1,2,3,4,5]| $\implies$ \verb|5|
  	\item construct an empty list
	  	\subitem \verb|Let list = []|
	\item prepend an item to a list
		\subitem \verb|[0] ++ [1,2,3,4,5]| $\implies$ \verb|[0,1,2,3,4,5]|
	\item append an item to a list
		\subitem \verb|[1,2,3,4,5] ++ [6]| $\implies$ \verb|[1,2,3,4,5,6]|
  \end{itemize}